% Options for packages loaded elsewhere
\PassOptionsToPackage{unicode}{hyperref}
\PassOptionsToPackage{hyphens}{url}
\documentclass[
]{book}
\usepackage{xcolor}
\usepackage{amsmath,amssymb}
\setcounter{secnumdepth}{5}
\usepackage{iftex}
\ifPDFTeX
  \usepackage[T1]{fontenc}
  \usepackage[utf8]{inputenc}
  \usepackage{textcomp} % provide euro and other symbols
\else % if luatex or xetex
  \usepackage{unicode-math} % this also loads fontspec
  \defaultfontfeatures{Scale=MatchLowercase}
  \defaultfontfeatures[\rmfamily]{Ligatures=TeX,Scale=1}
\fi
\usepackage{lmodern}
\ifPDFTeX\else
  % xetex/luatex font selection
\fi
% Use upquote if available, for straight quotes in verbatim environments
\IfFileExists{upquote.sty}{\usepackage{upquote}}{}
\IfFileExists{microtype.sty}{% use microtype if available
  \usepackage[]{microtype}
  \UseMicrotypeSet[protrusion]{basicmath} % disable protrusion for tt fonts
}{}
\makeatletter
\@ifundefined{KOMAClassName}{% if non-KOMA class
  \IfFileExists{parskip.sty}{%
    \usepackage{parskip}
  }{% else
    \setlength{\parindent}{0pt}
    \setlength{\parskip}{6pt plus 2pt minus 1pt}}
}{% if KOMA class
  \KOMAoptions{parskip=half}}
\makeatother
\usepackage{longtable,booktabs,array}
\usepackage{calc} % for calculating minipage widths
% Correct order of tables after \paragraph or \subparagraph
\usepackage{etoolbox}
\makeatletter
\patchcmd\longtable{\par}{\if@noskipsec\mbox{}\fi\par}{}{}
\makeatother
% Allow footnotes in longtable head/foot
\IfFileExists{footnotehyper.sty}{\usepackage{footnotehyper}}{\usepackage{footnote}}
\makesavenoteenv{longtable}
\usepackage{graphicx}
\makeatletter
\newsavebox\pandoc@box
\newcommand*\pandocbounded[1]{% scales image to fit in text height/width
  \sbox\pandoc@box{#1}%
  \Gscale@div\@tempa{\textheight}{\dimexpr\ht\pandoc@box+\dp\pandoc@box\relax}%
  \Gscale@div\@tempb{\linewidth}{\wd\pandoc@box}%
  \ifdim\@tempb\p@<\@tempa\p@\let\@tempa\@tempb\fi% select the smaller of both
  \ifdim\@tempa\p@<\p@\scalebox{\@tempa}{\usebox\pandoc@box}%
  \else\usebox{\pandoc@box}%
  \fi%
}
% Set default figure placement to htbp
\def\fps@figure{htbp}
\makeatother
\setlength{\emergencystretch}{3em} % prevent overfull lines
\providecommand{\tightlist}{%
  \setlength{\itemsep}{0pt}\setlength{\parskip}{0pt}}
\usepackage[]{natbib}
\bibliographystyle{apalike}
\usepackage{bookmark}
\IfFileExists{xurl.sty}{\usepackage{xurl}}{} % add URL line breaks if available
\urlstyle{same}
\hypersetup{
  pdftitle={Vrije Opdracht},
  pdfauthor={Twan van der Lee},
  hidelinks,
  pdfcreator={LaTeX via pandoc}}

\title{Vrije Opdracht}
\author{Twan van der Lee}
\date{2025-11-16}

\begin{document}
\maketitle

{
\setcounter{tocdepth}{1}
\tableofcontents
}
\chapter{Introductie}\label{introductie}

Op deze website vind je de portofolio van Twan van der Lee. Bij vragen contacteer \href{mailto:twan.vanderlee@student.hu.nl}{\nolinkurl{twan.vanderlee@student.hu.nl}}.

\chapter{Curriculum Vitae (CV)}\label{curriculum-vitae-cv}

\section{\texorpdfstring{\textbf{Profiel}}{Profiel}}\label{profiel}

Leergierig, doelgericht, gedisciplineerd en altijd opzoek naar antwoorden,
de beste omschrijving van mijzelf. Ik ben dagelijks bezig met het verbreden
van mijn kennis.
Mijn vrije tijd besteed ik aan sport, zo doe ik aan krachttraining en voetbal
ik al jaren. Ook hecht ik veel waarde aan het besteden van tijd met mijn
vrienden en naasten.

\section{\texorpdfstring{\textbf{Opleidingen}}{Opleidingen}}\label{opleidingen}

\subsubsection{\texorpdfstring{\textbf{\emph{St.~Bonifatiuscollege}}}{St.~Bonifatiuscollege}}\label{st.-bonifatiuscollege}

Hoger algemeen voortgezet onderwijs (HAVO)\\
- Vakkenpakket: Natuur, Techniek \& Gezondheid

\subsubsection{\texorpdfstring{\textbf{\emph{Hogeschool Utrecht}}}{Hogeschool Utrecht}}\label{hogeschool-utrecht}

Bachelor Life Sciences (2022-Heden)\\
- Specialisatie: Biomolecular Research\\
- GPA 7.8/10

Minor Applied Bioanalytical and Pharmaceutical Chemistry (2024-2025)\\
- Skills: HPLC, Massaspectrometrie

Minor Data Science for Biology (2025-2026)\\
- Skills: R, Bash

\section{\texorpdfstring{\textbf{Werkervaring}}{Werkervaring}}\label{werkervaring}

\subsubsection{\texorpdfstring{\textbf{\emph{Albert Heijn}}}{Albert Heijn}}\label{albert-heijn}

Allround supermarkt medewerker (2018-Heden)

\section{\texorpdfstring{\textbf{Skills}}{Skills}}\label{skills}

\begin{itemize}
\tightlist
\item
  Data-analyse (R en Bash)\\
\item
  PCR / qPCR\\
\item
  Gel-elektroforese\\
\item
  DNA- en RNA-isolatie\\
\item
  Western blotting\\
\item
  Celkweektechnieken\\
\item
  HPLC / massaspectrometrie
\end{itemize}

\chapter{Plan voor de toekomst}\label{plan-voor-de-toekomst}

\begin{enumerate}
\def\labelenumi{\arabic{enumi}.}
\item
  Over 1/2 jaar wil ik nog niet werken maar nog verder studeren. Hierna wil ik dat mijn droombaan aan de volgende punten voldoet. Ten eerst wil ik op een lab werken waar mijn werd een grote positieve inpact heeft op de samenleving, denk dan hierbij aan onderzoek tegen kanker of andere ziektes. Ook wil ik dat mijn droom baan een mooie balans heeft met op het lab werken en de data verwerken achter de computer.
\item
  Vanaf volgend jaar ben ik op zoek naar een stage in de proteomics. Skills dieik hiervoor al heb zijn praktijkervaring met HPLC en massaspectometrie.
\item
  Voor deze eventuele stage is het voor mij handig om de data van de MS goed te kunnen verwerken met R.
\end{enumerate}

\chapter{Vrije opdracht}\label{vrije-opdracht}

\section{\texorpdfstring{\textbf{\emph{Keuze nieuwe skill}}}{Keuze nieuwe skill}}\label{keuze-nieuwe-skill}

Voor het uitkiezen van een skill heb ik mijn mogelijke stage in de proteomics megenomen in de overweging. Aangezien ik al ervaring heb met het werken met HPLC-MS/MS lijkt mij het handig om te leren hoe ik met data verkregen uit de MS/MS een goede data analyse kan doen. Hierdoor heb ik gekozen om deze data verwerking met R te leren.

\section{\texorpdfstring{\textbf{\emph{Planning}}}{Planning}}\label{planning}

\begin{itemize}
\tightlist
\item
  4 uur Oriënteren en keuze maken.
\item
  4 uur R-omgeving en data klaar maken: R-setup, packages installeren, voorbeelddataset. binnenhalen\\
\item
  8 uur Leren en mini-proeven uitvoeren: Kernfuncties van de packages oefenen en begrijpen wat ze doen.\\
\item
  10 uur Toepassen op volledige dataset.\\
\item
  6 uur Uitwerken: portfolio, rapportage/samenvavtten
\end{itemize}

\section{\texorpdfstring{\textbf{\emph{Uiwerking}}}{Uiwerking}}\label{uiwerking}

\end{document}
